\documentclass[german]{latex4ei/latex4ei_sheet}



\title{Regelungstechnik}

\begin{document}

\section{Mathematische Grundlagen}
\begin{sectionbox}
    \subsection{komplexe Zahlen}

    \subsection{Dezibelrechnung}

    \subsection{Additionstheoreme}


    \subsection{Partialbruchzerlegung}


    \subsection{MNF / pq-Formel}


    \subsection{Polynomdivision}

    \subsection{Nullstellen}
    \begin{itemize}
        \item Ein Polynom der Ordnung $n$ hat $n$ Nullstellen.
        \item Sind die Koeffizienten reell, so sind die Nullstellen entweder reell oder komplex konjugiert.
    \end{itemize}
    
\end{sectionbox}

\section{Physikalische Grundlagen}
\begin{sectionbox}
    Newtonsche Gesetze:
    \begin{itemize}
        \item 1. Newtonsches Gesetz: $F = m \cdot a$ \\
        \item 2. Newtonsches Gesetz: $F = m \cdot \frac{dv}{dt}$ \\
        \item 3. Newtonsches Gesetz: $F = -F_{Gegenseite}$ \\
    \end{itemize}
\end{sectionbox}


\section{Formelzeichen}

Cosinusdarstellung:
    $a \cos(x) + b \sin(x) = R \cos(x - \phi)$

\section{Differentialgleichungen}
  


Auftereten von Nullstellen bei Polynom mit konstanten Koeffizienten:
\begin{itemize}
    \item reelle Nullstellen: $s_1, s_2, \ldots, s_n$ \\
    $P(s) = (s - s_1)(s - s_2) \ldots (s - s_n)$
    \item komplexe Nullstellen: $s_{n+1} = a + jb$ \\
    $P(s) = (s - s_{n+1})(s - s_{n+2}) \ldots (s - s_{n+m})$
\end{itemize}



\begin{sectionbox}{Allgemeine Form}
    \begin{align*}
        \frac{d^n y(t)}{dt^n} + a_{n-1} \frac{d^{n-1} y(t)}{dt^{n-1}} + \ldots + a_1 \frac{dy(t)}{dt} + a_0 y(t) \\= b_0 u(t) + b_1 \frac{du(t)}{dt} + \ldots + b_q \frac{d^q u(t)}{dt^q}
    \end{align*}
    \begin{itemize}
        \item $n$ = Ordnung der DGL $(q \leq n)$
        \item $a_i$ = Koeffizienten der DGL
        \item $b_i$ = Koeffizienten der Eingangsgröße
        \item $y(t)$ = Ausgangsgröße
        \item $u(t)$ = Eingangsgröße

\end{itemize}

Die Homogene Lösung ist die Lösung der DGL ohne Eingangsgröße:
\begin{align*}
    \frac{d^n y(t)}{dt^n} + a_{n-1} \frac{d^{n-1} y(t)}{dt^{n-1}} + \ldots + a_1 \frac{dy(t)}{dt} + a_0 y(t) = 0
\end{align*}
Durch Einsetzen der Lösung $y(t) = e^{\lambda t}$ erhält man die charakteristische Gleichung:
\begin{align*}
    \lambda^n + a_{n-1} \lambda^{n-1} + \ldots + a_1 \lambda + a_0 = 0
\end{align*}
Die Lösungen der charakteristischen Gleichung sind die Eigenwerte des Systems.
Die Eigenwerte sind die Nullstellen des Übertragungsfunktionsnenners/Polstellen der Übertragungsfunktion. \\
\\
Meistens sind inhhomogene DGLs höherer Ordnung gegeben.
Diese können gelöst werden durch:
\begin{itemize}
    \item Variation der Konstanten
    \item Ansatz der rechten Seite
    \item Laplace-Transformation
\end{itemize}
\end{sectionbox}


\section{Systemeigenschaften}
\begin{sectionbox}
    \subsection{Linearität}
    \vspace{1mm}
    \begin{itemize}
        \item Superposition: $S\{u_1(t) + u_2(t)\} = S\{u_1(t)\} + S\{u_2(t)\}$
        \item Homogenität: $S\{k \cdot u(t)\} = k \cdot S\{u(t)\}$
    \end{itemize}
    \subsection{Zeitinvarianz}
    \vspace{1mm}
    \begin{itemize}
        \item Ein System S ist zeitinvariant, wenn für jede beliebige Zeitverschiebung $τ$ gilt:
        $y(t −τ) = S\{u(t −τ)\}$
    \end{itemize}
    \subsection{Kausalität}
    \vspace{1mm}
    \begin{itemize}
        \item Ein System $S$ ist kausal, wenn der Wert des Eingangs $u(\bar{t})$ zum Zeitpunkt $\bar{t}$ keinen
        Einfluß auf den Ausgang $y(t)$ für $t < \bar{t}$ hat. \\
        Bzw, wenn im Eingang höhere Ableitungen als im Ausgang vorkommen.
    \end{itemize}


    
\end{sectionbox}

\section{Güteanforderungen}
\begin{sectionbox}
        \subsection{Stabilität}
    \vspace{1mm}
    \begin{itemize}
        \item asymptotisch stabil: alle Lösungen der charakteristischen Gleichung haben negative Realteile
        \item stabil: alle Lösungen der charakteristischen Gleichung haben Realteile $\leq 0$
        \item instabil: mindestens eine Lösung der charakteristischen Gleichung hat einen positiven Realteil
    \end{itemize}

    \subsection{Stationäre Genauigkeit}
    \vspace{1mm}
    \begin{itemize}
        \item $e_{ss} = \lim_{t \to \infty} e(t)$
        \item $e_{ss} = 0$ für alle stationären Eingangsgrößen
    \end{itemize}
    \subsection{Dynamisches Verhalten}
    \vspace{1mm}
    \begin{itemize}
        \item Überschwingen: $e_{max} = \max\limits_{t \geq 0} e(t)$
        \item Einschwingzeit: $t_e = t_{max} - t_{ss}$
        \item Anstiegszeit: $t_a = t_{ss} - t_0$
        \item Regelzeit: $t_r = t_{ss} - t_{min}$
    \end{itemize}
    \subsection{Robustheit}
    \vspace{1mm}
    Forderungen 1 - 3 werden trotz Unsicherheiten erfüllt!
    \begin{itemize}
        \item Unsicherheiten bewegen sich innerhalb vorgegebener Grenzen
        \item Unsicherheiten → Unterschiede zwischen Modell u. Wirklichkeit
        \item Ursache von Unsicherheiten:
        \begin{itemize}
            \item Modell beschreibt Wirklichkeit nur annähernd und vereinfacht
            \item Regelstrecke verändert sich (Toleranzen, Alterung, Verschleiß, ...)
        \end{itemize}
    \end{itemize}
    
\end{sectionbox}
    \begin{sectionbox}
        \subsection{Modell des Standardregelkreises}
        \vspace{1mm}
        Führungsübertragungsfunktion $G_{W}(s)$ und Störgrößen- übertragungsfunktion $G_{Z}(s)$ müssen zusammen 1 ergeben:
        \begin{align*}
            G_{W}(s) + G_{Z}(s) = 1
        \end{align*}


        \subsection{Stationäres Verhalten}
        \vspace{1mm}
        Bleibende Regelabweichung: $e_{ss} = \lim_{t \to \infty} e(t)$ \\
        Lässt sich im Nenner von $G(s)$ mindestens ein $s$ ausklammern, ist die Bleibende Regelabweichung $0$.
    \end{sectionbox}


    \begin{sectionbox}
        \textbf{Stabilitäts-/Beiwertbedingungen:}
     %   Vorzeichenbedingung für Ordnung $n > 2$ (nur notwendig):
     %   System ist asymptotisch stabil $⇒$ alle Koeff. ai , i = 0,1, . . . ,n besitzen gleiches
     %   Vorzeichen, aber nur für Systeme mit $n ≤ 2$.
     %   Für Systeme mit $n > 2$ ist die Vorzeichenbedingung nicht ausreichend, d.h. es müssen
     %   zusätzliche Beiwertebedingungen oder das Hurwitz-Kriterium erfüllt sein.
        
     %   Bedingung für Systeme mit n = 3:
     %   System ist asymptotisch stabil ⇒ VZB ist erfüllt und an > 0 und a0a3−a1a2 < 0

    %    Bedingung für Systeme mit n = 4:
    %        System ist asymptotisch stabil ⇒ VZB ist erfüllt und an > 0 und
    %    a4a2
    %    1+a0a2
    %    3−a1a2a3 < 0

    \begin{itemize}
        \item \textbf{Vorzeichenbedingung} (notwendig):
        System ist asymptotisch stabil $\Rightarrow$ alle Koeffizienten $a_i$, $i = 0,1, \ldots, n$ besitzen gleiches Vorzeichen. Für Systeme mit $n \leq 2$ auch hinreichend.
        \item \textbf{Beiwertbedingungen} (hinreichend) für Systeme mit $n > 2$: Determinante der $n \times n$ Hurwitz-Matrix muss $|\Delta_n| > 0$ sein. Für $|\Delta_n| = 0$ grenzstabil.
    \end{itemize}

    \textbf{Hurwitz-Matrix:}\\
    \[
        %\Delta_n =
        \begin{bmatrix}
        a_{n-1} & a_{n-3} & a_{n-5} & \cdots & 0 & 0 & 0 \\
        a_n & a_{n-2} & a_{n-4} & \cdots & 0 & 0 & 0 \\
        0 & a_{n-1} & a_{n-3} & \cdots & 0 & 0 & 0 \\
        0 & 0 & a_n & a_{n-2} & \cdots & 0 & 0 \\
        0 & 0 & 0 & a_{n-1} & \cdots & 0 & 0 \\
        \vdots & \vdots & \vdots & \vdots & \ddots & \vdots & \vdots \\
        0 & 0 & 0 & 0 & \cdots & a_3 & a_1 \\
        0 & 0 & 0 & 0 & \cdots & a_4 & a_2 & a_0
        \end{bmatrix}
    \]

    \end{sectionbox}



    \subsection{PID-Regler}
    \begin{sectionbox}
        \begin{align*}
            u(t) = K_R\{e(t)\} = k_p e(t) + k_i \int_0^\tau e(t) d\tau + k_d \frac{de(t)}{dt} \\
        \end{align*}
        \begin{itemize}
            \item $k_p$ = Proportionalbeiwert
            \item $k_i$ = Integralbeiwert
            \item $k_d$ = Differentialbeiwert
        \end{itemize}
\end{sectionbox}


\section{Frequenzgang}
\begin{sectionbox}
    \[
        G(j\omega) = \frac{Y(j\omega)}{U(j\omega)} = \frac{V}{s^{\lambda}} \frac{p(s)}{q(s)}
    \]
    \begin{itemize}
        \item $V$ ... statischer Verstärkungsfaktor
        \item $λ$ ... ganzzahlig ($+$ zählt Polstellen, $−$ Nullstellen)
        \item $p(s),q(s)$ sind die Polynome in $s$. $p(0)=q(0) = 1$
    \end{itemize}
    \vspace{2mm}

    \textbf{Betragsverlauf:}
    \begin{center} 
        $|G(j\omega)| = |V|_{dB} - |(j\omega)^{\lambda}|_{dB} + |p(j\omega)|_{dB} - |q(j\omega)|_{dB}$
    \end{center}

    \textbf{Phasenverlauf:}
    \begin{center} 
        $\arg G(j\omega) = \arg V - \lambda \arg(j\omega) + \arg p(j\omega) - \arg q(j\omega)$
    \end{center}

    \vspace{3mm}
    \textbf{4 elementare Bestandteile:}
    \begin{enumerate}
    \item \textbf{Verstärkungsfaktor} \(V\) (reell).
    
    \item \textbf{Potenzfaktor} \((j\omega)^{\lambda}\) 
    \[
    \implies \pm 20\lambda \, \text{dB/Dek} \quad \text{und} \quad \pm 90\lambda^\circ.
    \]

    \item \textbf{Linearfaktor} \(\left( 1 + \frac{s}{\omega_k} \right)^{\pm 1}\) 
    \[
    \implies \text{Knick bei } \omega_k, \, \text{Steigung } \pm 20 \, \text{dB/Dek}.
    \]

    \item \textbf{Quadratischer Faktor} \(\left( 1 + 2\zeta \frac{s}{\omega_k} + \left(\frac{s}{\omega_k}\right)^2 \right)^{\pm 1}\) 
    \[
    \implies \text{Steigung } \pm 40 \, \text{dB/Dek}, \, \text{Phase } 0 \to \pm 180^\circ.
    \]
\end{enumerate}

\end{sectionbox}


\end{document}